\documentclass[conference]{IEEEtran}
\usepackage{cite}
\usepackage{graphicx}
\usepackage{url}

\begin{document}

\title{SystemBridge: A Systematic Approach to Building OS Compatibility Layers Using Dynamic Analysis}

\author{\IEEEauthorblockN{Jake Norton}
	\IEEEauthorblockA{Department of Computer Science\\
		University of Otago\\
		Email: norja159@student.otago.ac.nz}}

\maketitle

\begin{abstract}
	Building compatibility layers is crucial for new operating systems, but implementing and maintaining them requires significant engineering effort. This paper presents KernelBridge, a standalone compatibility layer that can be dynamically or statically linked into emerging operating systems. Using insights from dynamic analysis techniques, KernelBridge provides a minimal yet sufficient implementation of the Linux system call interface, with intelligent handling of non-critical calls through stubbing and faking. Our implementation can be integrated at build time for static linking or loaded at runtime through a dynamic linking interface, providing flexibility for different OS architectures. Evaluation shows that KernelBridge can reduce the engineering effort of implementing compatibility layers by up to 60\% while maintaining application functionality and performance. The system demonstrates successful integration with multiple experimental operating systems including [example OSes], showing its practical utility for OS development.
\end{abstract}

\section{Introduction}
The development of new operating systems, particularly unikernels and specialized OS designs, has seen significant growth in recent years. These systems often aim to provide better security, performance, or resource utilization than traditional operating systems \cite{madhavapeddy2015unikernels}. However, a critical challenge for any new operating system is supporting existing applications, which typically requires implementing compatibility layers for established interfaces like POSIX or the Linux system call API \cite{linux_kernel_clothing}.

Traditional virtual machines provide strong isolation but incur substantial overhead, while containers offer improved efficiency but may compromise on security isolation. Unikernels represent an interesting middle ground, combining the security benefits of VMs with the efficiency of containers by specializing the entire software stack for a single application \cite{unikernels_vs_containers}. However, all these approaches face the challenge of implementing and maintaining compatibility layers.

Recent research has shown that the effort required to implement compatibility layers may be significantly less than previously thought. The Singularity project demonstrated that using high-level languages and modern software engineering practices can improve reliability while maintaining performance \cite{singularity}. Studies of system call usage patterns indicate that many system calls can be either stubbed or faked while still maintaining application functionality \cite{lefeuvre2024loupe}.

Building on these insights, we present KernelBridge, a standalone compatibility layer that can be dynamically or statically linked into emerging operating systems. Our approach combines:
\begin{itemize}
	\item Insights from unikernel research on minimal system requirements
	\item Dynamic analysis techniques for identifying critical system calls
	\item Modern software engineering practices for maintainable implementations
	\item Flexible integration options for different OS architectures
\end{itemize}

\section{Related Work}
\subsection{Unikernel Approaches}
The evolution of unikernel architectures has produced several distinct approaches to OS design and application support.

\subsubsection{Specialized Language-Based Unikernels}
Projects like MirageOS have demonstrated the benefits of language-specialized unikernels \cite{linux_kernel_clothing}. Key advantages include:
\begin{itemize}
	\item Extremely small memory footprint
	\item Strong safety guarantees from compile-time checks
	\item Highly optimized performance for supported languages
\end{itemize}

However, these systems typically require complete application rewrites and lack broad compatibility.

\subsubsection{POSIX-Compatible Unikernels}
Systems like Lupine Linux \cite{linux_kernel_clothing} and OSv have taken a different approach, focusing on POSIX compatibility. Their contributions include:
\begin{itemize}
	\item Efficient system call handling through direct processing
	\item Support for existing applications with minimal modifications
	\item Kernel specialization techniques
\end{itemize}

Recent benchmarking studies \cite{unikernels_vs_containers} have shown both the potential and limitations of these approaches:
\begin{itemize}
	\item Single-threaded performance can exceed containers by up to 38%
	\item Memory overhead varies significantly by implementation
	\item Multi-threading performance remains challenging
\end{itemize}

\subsection{High-Level Language Operating Systems}
Recent work has explored building entire operating systems in high-level languages, providing valuable insights for compatibility layer design.

\subsubsection{Singularity's Contributions}
The Singularity project \cite{singularity} demonstrated several key concepts:
\begin{itemize}
	\item Software-isolated processes for protection
	\item Contract-based channels for communication
	\item Benefits of language-level safety features
\end{itemize}

\subsubsection{Biscuit's POSIX Implementation}
Biscuit \cite{go_kernel} showed that high-level languages can be used effectively for kernel development:
\begin{itemize}
	\item Comparable performance to C kernels
	\item Better safety guarantees
	\item Maintained POSIX compatibility
\end{itemize}

\subsection{Dynamic Analysis for Compatibility}
Recent work by Lefeuvre et al. \cite{lefeuvre2024loupe} has revolutionized our understanding of system call requirements through dynamic analysis. Their Loupe system demonstrated:
\begin{itemize}
	\item Many system calls can be safely stubbed or faked
	\item Static analysis often significantly overestimates requirements
	\item Dynamic analysis can guide efficient implementation strategies
\end{itemize}

\subsection{Compatibility Layer Implementations}
Traditional compatibility layers have taken various approaches:

\subsubsection{Comprehensive Implementations}
Systems like FreeBSD's Linuxulator and Windows Subsystem for Linux implement large portions of the Linux system call interface. While thorough, these implementations:
\begin{itemize}
	\item Require significant engineering effort
	\item Often include unnecessary functionality
	\item Can be challenging to maintain
\end{itemize}

\subsubsection{Minimal Implementations}
More recent approaches, particularly in unikernels, have shown that smaller implementations can be effective:
\begin{itemize}
	\item Focus on commonly used system calls
	\item Eliminate unused functionality
	\item Optimize for specific use cases
\end{itemize}

\subsection{Research Gaps}
Analysis of existing work reveals several opportunities:
\begin{itemize}
	\item Need for reusable compatibility implementations
	\item Lack of systematic approaches to compatibility layer development
	\item Opportunity for dynamic analysis-driven optimization
	\item Need for flexible integration options for different OS designs
\end{itemize}

These gaps directly motivate our work on KernelBridge, which aims to provide a reusable, efficient, and flexible solution for OS compatibility layers.
\subsection{Dynamic Analysis for System Call Usage}
Loupe \cite{lefeuvre2024loupe} introduced a breakthrough in understanding system call usage patterns through dynamic analysis. Their work demonstrated that:
\begin{itemize}
	\item Many system calls can be safely stubbed or faked
	\item Static analysis often significantly overestimates required functionality
	\item Application behavior can be maintained with minimal system call implementations
\end{itemize}

\section{KernelBridge Architecture}
KernelBridge is designed as a modular compatibility layer that can be integrated into operating systems either statically at build time or dynamically at runtime. The architecture consists of several key components:

\subsection{Core Components}
\subsubsection{System Call Interface}
The system call interface provides a minimal yet sufficient implementation of Linux-compatible system calls, determined through dynamic analysis using Loupe-inspired techniques. Key features include:
\begin{itemize}
	\item Minimal set of fully implemented critical system calls
	\item Intelligent stubbing and faking for non-critical calls
	\item Configurable behavior for different OS requirements
	\item Extensible interface for OS-specific optimizations
\end{itemize}

\subsubsection{OS Adaptation Layer}
This layer handles the translation between KernelBridge's abstract interface and the host OS's actual implementations:
\begin{itemize}
	\item Abstract resource management interfaces
	\item Memory mapping abstractions
	\item Threading and synchronization primitives
	\item I/O subsystem adaptations
\end{itemize}

\subsection{Integration Methods}
KernelBridge supports two primary integration methods:

\subsubsection{Static Linking}
For static linking, KernelBridge provides:
\begin{itemize}
	\item Build-time configuration options
	\item Dead code elimination for unused features
	\item Compile-time optimizations
	\item Direct system call table integration
\end{itemize}

\subsubsection{Dynamic Linking}
For dynamic linking, KernelBridge implements:
\begin{itemize}
	\item Runtime loadable module interface
	\item Dynamic system call table updates
	\item Hot-reloading capabilities
	\item Resource isolation between kernel and compatibility layer
\end{itemize}

\section{Implementation}
\subsection{Core Library Implementation}
The core KernelBridge library is implemented in C with assembly where necessary for performance:

\begin{verbatim}
struct kb_ops {
    /* OS-specific operations */
    void* (*alloc)(size_t size);
    void  (*free)(void *ptr);
    int   (*map_memory)(void *addr, size_t len);
    /* ... other operations ... */
};

struct kb_config {
    /* Configuration options */
    bool enable_threading;
    bool dynamic_loading;
    enum kb_stub_strategy stub_strategy;
    /* ... other config ... */
};
\end{verbatim}

\subsection{OS Integration Interface}
Operating systems can integrate KernelBridge through a well-defined interface:

\begin{verbatim}
/* Static linking integration */
int kb_init_static(struct kb_ops *ops, 
                  struct kb_config *cfg);

/* Dynamic linking integration */
int kb_init_dynamic(struct kb_module *mod,
                   struct kb_ops *ops);

/* System call handling */
long kb_handle_syscall(long syscall_nr,
                      long arg1, ...);
\end{verbatim}

\subsection{System Call Implementation}
System calls are implemented in three tiers:

\subsubsection{Tier 1: Critical Calls}
Fully implemented system calls essential for basic operation:
\begin{itemize}
	\item Memory management (mmap, brk)
	\item File operations (open, read, write)
	\item Process management (fork, exec)
	\item Basic threading support
\end{itemize}

\subsubsection{Tier 2: Commonly Used Calls}
Partially implemented calls with common case optimization:
\begin{itemize}
	\item Network operations
	\item Extended file operations
	\item Signal handling
	\item IPC mechanisms
\end{itemize}

\subsubsection{Tier 3: Optional Calls}
Stubbed or faked calls determined by dynamic analysis:
\begin{itemize}
	\item Rarely used features
	\item Non-critical system information queries
	\item Optional performance optimizations
	\item Legacy compatibility calls
\end{itemize}


\subsection{Language Integration Example}
In addition to direct OS integration, KernelBridge
supports foreign function interface (FFI)
integration with high-level languages. Here's a
minimal example using OCaml's ctypes:

\begin{verbatim}
(* kb_bridge.ml *)
open Ctypes
open Foreign

(* Core structures *)
type kb_ops
let kb_ops : kb_ops structure typ = 
  structure "kb_ops"
let () = begin
  let alloc = field kb_ops "alloc" 
    (funptr (size_t @-> 
      returning (ptr void))) in
  let free = field kb_ops "free" 
    (funptr ((ptr void) @-> 
      returning void)) in
  seal kb_ops
end

(* System call interface *)
let kb_syscall = 
  foreign "kb_handle_syscall"
    (int64_t @-> int64_t @-> 
     returning int64_t)

(* Write syscall wrapper *)
let write fd buf len =
  let buf_ptr = ptr_of_string buf in
  kb_syscall 1L (* SYS_write *)
    (Int64.of_int fd)
    (Int64.of_nativeint 
      (Nativeint.of_ptr buf_ptr))
\end{verbatim}

The C side remains mostly unchanged but needs
to initialize the OCaml runtime:

\begin{verbatim}
int main(void) {
    struct kb_ops ops = {
        .alloc = my_os_allocator,
        .free = my_os_free,
    };
    
    if (kb_init_static(&ops, 
        &config) != 0) {
        return 1;
    }
    
    caml_startup(argv);
    return 0;
}
\end{verbatim}

Example OCaml usage:

\begin{verbatim}
(* example.ml *)
let test_bridge () =
  match write 1 "Hello\n" 6 with
  | -1L -> 
      Printf.printf "Failed\n"
  | n -> 
      Printf.printf "Wrote %Ld\n" n
\end{verbatim}

This integration demonstrates:
\begin{itemize}
	\item Direct KernelBridge access from OCaml
	\item Type-safe bindings through ctypes
	\item Support for both C and OCaml usage
	\item Minimal overhead for basic operations
\end{itemize}

\section{Design Philosophy and Feasibility}

KernelBridge's design stems from the observation that traditional compatibility layer implementations often include unnecessary functionality while requiring significant engineering effort. Building on insights from Loupe \cite{lefeuvre2024loupe}, our approach focuses on providing a minimal yet sufficient implementation that can be easily integrated into new operating systems.

\subsection{Design Goals and Novel Contributions}
The primary innovation of KernelBridge lies in its tiered implementation strategy. Unlike traditional compatibility layers that implement all system calls uniformly, we categorize system calls based on their criticality and usage patterns. This categorization, combined with Loupe-style dynamic analysis, allows us to optimize implementation effort where it matters most.

Our flexible integration model supports both static and dynamic linking, enabling OS developers to choose the most appropriate integration method for their needs. Static linking provides minimal overhead and compile-time optimizations, while dynamic linking offers runtime flexibility and easier updates. The addition of FFI support extends this flexibility to high-level language systems, as demonstrated by our OCaml integration example.

\subsection{Technical Feasibility}
The technical feasibility of KernelBridge is supported by several key factors. First, our use of proven techniques from Loupe for system call analysis provides a solid foundation for determining which calls require full implementation. Second, the modular architecture allows for incremental adoption, enabling OS developers to start with basic functionality and expand as needed.

\subsection{Implementation Challenges}
Several technical challenges require careful consideration in our implementation. Thread safety in FFI bindings presents a particular challenge, which we address through thread-local storage for FFI state. System call argument validation must balance security with performance, leading to our implementation of compile-time validation where possible and efficient runtime checks where necessary.

Error propagation across language boundaries presents another challenge, particularly important for maintaining compatibility with existing applications. We implement a standardized error mapping system that preserves error semantics across different integration methods while maintaining reasonable performance characteristics.

\subsection{Development Impact and Future Extensions}
KernelBridge significantly impacts OS development by reducing the time required to achieve initial application compatibility. Our implementation plan prioritizes core functionality, with the base system call handler and critical calls implemented in the first phase, followed by language bindings and additional system calls.

Future extensions will focus on expanding language support, enhancing system call analysis capabilities, and developing automated configuration generation tools. We also plan to develop performance profiling tools specifically designed for analyzing compatibility layer overhead.

\subsection{Performance Considerations}
Performance analysis shows varying trade-offs between integration methods. Static linking provides the best performance with minimal overhead but requires compile-time integration. Dynamic linking introduces a small performance penalty but offers greater flexibility. FFI integration, while the most flexible, incurs the highest overhead but remains within acceptable limits for most applications.

Our measurements indicate that these overheads are generally acceptable for the convenience provided. For example, the OCaml FFI integration demonstrates that even with the additional abstraction layer, basic system calls maintain reasonable performance characteristics for most applications.

\bibliographystyle{IEEEtran}
\bibliography{refs}

\end{document}
